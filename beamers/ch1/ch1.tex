% =========================================================================
%
%
% =========================================================================
\documentclass[aspectratio=1610]{beamer}

% ========================= Theme =========================================
\usetheme{CambridgeUS}
\usecolortheme{default}

% ========================= Packages ======================================
\usepackage{tikz}
\usepackage{graphicx}
\usepackage{graphics}
\usepackage{pgfplots}
\pgfplotsset{width=7cm,compat=1.17}

% ========================= Infor on authors ==============================
\title{Intro to Python --- SMM692}
\subtitle{Organization of the Module}
\author{Simone Santoni}
\institute{Bayes Business School}
\date{2022}

% ========================= Infor on authors ==============================
\definecolor{base_c}{rgb}{0.6,0,0}
\definecolor{comp_c}{rgb}{0.09803921568627451, 0.6901960784313725, 0.7529411764705882}
\definecolor{tri_1}{rgb}{0.09803921568627451, 0.7686274509803922, 0.19215686274509805}
\definecolor{tri_2}{rgb}{0.19215686274509805, 0.09803921568627451, 0.7686274509803922}

% ========================= Document  ====================================
\begin{document}

\frame{\titlepage}

% ------------------------- Background -----------------------------------

\section{Background}

\subsection{Justification for SMM692}

\begin{frame}{Why Python?}
	\begin{columns}
		\begin{column}{0.5\textwidth}
			\begin{itemize}
				\item Python is a general-purpose, high-level programming language
				\begin{itemize}
					\item Traditionally, Python was used for developing desktop and web applications
				\end{itemize}
				\item So, why should business analytics (BA)professionals learn it?
				\begin{itemize}
				\item Python is the center of one of the richest --- if not the richest --- data-science  ecosystem
				\item Python's popularity surge is related with the emergence and development of the data science field 
			        \end{itemize}
		        \end{itemize}
               \end{column}
	       \begin{column}{0.5\textwidth}
		\begin{figure}
			\centering
			\begin{tikzpicture}[scale=1]
				\begin{axis}[
					title = Interest Over Time,
					xtick={0, 47, 95, 143},
					xticklabels={2010, 2014, 2018, 2022},
					xlabel = Time,
					ylabel = Google Trends Index,
					legend pos=north west
					]
				    \addplot[
					scatter,only marks,scatter src=explicit symbolic,
					scatter/classes={
					    python={mark=*,draw=base_c, fill=base_c!20},
					    r={mark=*,draw=tri_1, fill=tri_1!20},
					    julia={mark=*,draw=tri_2,fill=tri_2!20}
					}
				    ]
				    table[x=,y=y,meta=label]{data/google_trends.dat};
				    \legend{Python, R, Julia}
				\end{axis}
				\end{tikzpicture}
		\end{figure}
	       \end{column}
        \end{columns}
\end{frame}

\begin{frame}{Why an Introductory Module on Python in a BA Post-Grad Course?}
	\ldots
\end{frame}

% ------------------------- Scope ----------------------------------------
\subsection{Scope of SMM692}

\begin{frame}{Python for What?}
\ldots	
\end{frame}

\begin{frame}{Python in the Business Analytics Sphere}
\ldots
\end{frame}

\begin{frame}{SMM692 Foci}
	\ldots
\end{frame}

% ------------------------- Teaching and learning activities --------------
\section{Learning/Teaching Actitivies}

\subsection{Phylosophy Behind SMM692}

\begin{frame}{}
\ldots
\end{frame}

% ------------------------- Assessment  -----------------------------------
\section{Assessment}

\begin{frame}{}
\ldots
\end{frame}

\begin{frame}{Hello there}
	...
\end{frame}

\end{document}