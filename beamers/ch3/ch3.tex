% !TeX TXS-program:compile = txs:///pdflatex/[--shell-escape]

% =========================================================================
\documentclass[aspectratio=1610]{beamer}

% ========================= Theme =========================================
\usetheme{CambridgeUS}
\usecolortheme{default}

% ========================= Essential packages ============================
\usepackage{hyperref}

% ========================= Plotting ======================================
\usepackage{calc}
\usepackage{tikz}
\usetikzlibrary{arrows,
                arrows.meta,
                calc,
		chains,
                quotes,
                positioning,
		shapes,
                shapes.geometric}
\usepackage{graphicx}
\usepackage{graphics}
\usepackage{pgfplots}
\pgfplotsset{width=7cm,compat=1.17}

% ========================= Infor on authors ==============================
\title{Intro to Python --- SMM692}
\subtitle{Python Objects}
\author{Simone Santoni}
\institute{Bayes Business School}
\date{MSc Pre-Course Series}

% ============================ Colors =====================================
\definecolor{base_c}{rgb}{0.6,0,0}
\definecolor{comp_c}{rgb}{0.09803921568627451, 0.6901960784313725, 0.7529411764705882}
\definecolor{tri_1}{rgb}{0.09803921568627451, 0.7686274509803922, 0.19215686274509805}
\definecolor{tri_2}{rgb}{0.19215686274509805, 0.09803921568627451, 0.7686274509803922}

% ========================= TOC  ==========================================
\AtBeginSubsection[]
{
    \begin{frame}
        \frametitle{Outline}
        \tableofcontents[currentsection,currentsubsection]
    \end{frame}
}

% ========================= Document  ====================================
\begin{document}

\begin{frame}
	\titlepage
\end{frame}

\begin{frame}{Outline}
	\tableofcontents
\end{frame}

% ------------------------- Background -----------------------------------

\section{The Chapter in a Nutshell}

\begin{frame}
    \frametitle{Scope}
    In Chapter 3, the attention revolves around the following topics:

    \begin{itemize}
        \item The concept of Python object 
        \item The types of Python objects
        \item The characteristics of each Python object
    \end{itemize}
\end{frame}

\begin{frame}
    \frametitle{Why Shall I Learn About Python Objects?}
    \begin{itemize}
    	\item Built-in objects make coding efficient and easy
        \begin{itemize}
            \item  For example, using the \href{https://docs.python.org/3/tutorial/introduction.html\#strings}{string} object, we can represent and manipulate a piece of text --- e.g., a newspaper article --- without loading any \href{https://docs.python.org/3/tutorial/modules.html}{module}
        \end{itemize}
           	\item Built-in objects are flexible
            \begin{itemize}
                \item For example, we can deploy built-in objects to create a \href{https://docs.python.org/3/tutorial/classes.html}{class}
            \end{itemize}
    	\item Built-in objects have been created and refined over time by a large community of expert developers. Hence, they are  often  more  efficient  than  ad-hoc objects (unless the creator of the ad-hoc object knows her business!)
    \end{itemize}
\end{frame}

\begin{frame}
    \frametitle{Learning Goals}
    At the end of the chapter, you will be able to evaluate the various types of Python objects regarding:
    \begin{itemize}
        \item Key features
        \item Use cases/roles
        \item Available methods
    \end{itemize}
\end{frame}

\section{Python Objects Fundamentals}

\begin{frame}
    \frametitle{What is a Python Object?}
    
    In essence, Python objects are pieces of data. Mark Lutz, the author of the popular book \href{https://www.google.co.uk/books/edition/Learning_Python/4pgQfXQvekcC?hl=en&gbpv=0}{\textcolor{blue}{Learning Python}}, points out

    \vspace{2em}
    
    \begin{quote}
    \textit{``... in Python, we do things with stuff. ``Things'' take the form of operations like addition and concatenation, and ``stuff'' refers to the objects on which we perform those operations''}
    \end{quote}

\end{frame}

\begin{frame}
    \frametitle{What Are the Main Families of Python Objects?}
    In Python, there are two families of objects:
    
    \begin{itemize}
        \item Built-in objects provided by the Python language itself
        \item Ad-hoc objects --- called \href{https://docs.python.org/3/tutorial/classes.html}{classes} --- we can create to accomplish specific goals
    \end{itemize}
    
    \end{frame}

\begin{frame}
    \frametitle{What Are the Main Types of Built-In Python Objects?}
    \begin{itemize}
        \item Strings
        \item Numbers
        \item Data containers
        \begin{itemize} 
            \item Lists 
            \item Dictionaries
            \item Tuples 
            \item Sets 
        \end{itemize}
        \item Files 
        \item Python statements, syntax, and control flow
        \item Iterators 
    \end{itemize}
\end{frame}

\section{Built-In Python Object Types }
\subsection{Numbers \& Strings}

\begin{frame}
    \frametitle{Number Type Fundamentals}
    \begin{columns}
        \begin{column}{0.5\textwidth}
            The most common number types are integers and floating-point numbers:
            \begin{itemize}
                \item Integers are whole numbers such as 0, 4, or -12
                \item Floating-point numbers represent real numbers such as 0.5, 3.1415, or -1.6e-19
                \begin{itemize}
                    \item However, floating points in Python do not have --- in general
                    --- the same value as the real number they represent
                    \item It is worth noticing that any single number with a period `.'
                    is considered a floating point in Python 
                \end{itemize}
            \end{itemize}
        \end{column}
        \begin{column}{0.5\textwidth}
            \begin{minted}{python}
# Print a string object
print("Bazinga")
# Print the result of an algebraic operation
print(2 + 4)
	        \end{minted}
		\rule{\textwidth}{1pt}
        \end{column}
    \end{columns}
\end{frame}

\begin{frame}
    \frametitle{String Type Fundamentals}
\end{frame}

\subsection{Data Containers}

\begin{frame}
    \frametitle{List Type Fundamentals}
\end{frame}
\begin{frame}
    \frametitle{Dictionary Type Fundamentals}
\end{frame}

\begin{frame}
    \frametitle{Tuple Type Fundamentals}
\end{frame}

\begin{frame}
    \frametitle{Set Type Fundamentals}
\end{frame}

\subsection{Files}

\begin{frame}
    \frametitle{...}
\end{frame}

\subsection{Python Statements, Syntax, and Control Flow}

\begin{frame}
    \frametitle{...}
\end{frame}

\subsection{Iterators }

\begin{frame}
    \frametitle{...}
\end{frame}

\section{Wrap-Up}

\begin{frame}
    \frametitle{At the End of the Chapter, You Will Be Able to...}
\end{frame}

\end{document}