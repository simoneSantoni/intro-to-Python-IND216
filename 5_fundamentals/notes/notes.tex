%% =========================== Document class =============================
\documentclass[a4paper,11pt]{article}
% !TeX TXS-program:compile = txs:///pdflatex/[--shell-escape]

%% ========================= Essential packages ===========================
\usepackage[utf8]{inputenc}
\usepackage{graphicx}
\usepackage[utf8]{inputenc}
\usepackage[margin=2cm]{geometry}   
%\usepackage{color}
\usepackage[dvipsnames]{xcolor}
\usepackage{tcolorbox}
\usepackage{paracol} % added <<<<<<<<<<<<<<<<<<<<<
\usepackage{lineno}
\usepackage{amsmath}
\tcbuselibrary{minted,skins,breakable}

%% =========================== Color options ==============================
% custom colors
\definecolor{base_c}{rgb}{0.6,0,0}
\definecolor{comp_c}{rgb}{0.09803921568627451, 0.6901960784313725, 0.7529411764705882}
\definecolor{tri_1}{rgb}{0.09803921568627451, 0.7686274509803922, 0.19215686274509805}
\definecolor{tri_2}{rgb}{0.19215686274509805, 0.09803921568627451, 0.7686274509803922}

%% =============================== Links ==================================
\usepackage[
colorlinks=true,
allcolors=base_c,
%citecolor=CadetBlue,
%urlcolor=CadetBlue
]{hyperref}%

% =========================== Utilities ===================================
\usepackage{lipsum}% generate filler text

%% ========================== Code input ==================================
\newtcblisting{pythoncode}[2][]{
  listing engine=minted,
  breakable,
  colback=green,
  colframe=black!70,
  listing only,
  minted style=tango,
  minted language=python,
  minted options={numbersep=3mm,texcl=true,#1},
  left=5mm,enhanced,
  overlay={\begin{tcbclipinterior}\fill[black!25] (frame.south west)
            rectangle ([xshift=5mm]frame.north west);\end{tcbclipinterior}},
            #2,
}

% ========================= Cornell Notes stuff ============================
% question
\newcommand{\question}[1]{% Ask the question
    \begin{tcolorbox}[colback=comp_c!10,colframe=comp_c,sidebyside align=top,width=\linewidth,before skip=1ex]
        #1
    \end{tcolorbox}%    
    \switchcolumn% now write in the right column   
}

% note
\newcommand{\note}[1]{% Add as many notes as you like
    \begin{tcolorbox}[colback=white!0,colframe=white!10,width=\linewidth,before skip=1ex]
        #1
    \end{tcolorbox}         
}   

% summary
\newcommand{\summary}[2][]{%
\begin{minipage}[b]{\textwidth}
    \vspace*{\baselineskip}
    \begin{tcolorbox}[colframe=tri_2!75,fonttitle=\large\bfseries\sffamily,
        after skip = \baselineskip,
        title=Summary]
        #2
    \end{tcolorbox}
\end{minipage}
#1}

% proportions
\setcolumnwidth{0.30\textwidth/20pt,0.70\textwidth}% column separation =20pt
\setlength{\columnseprule}{2pt} % column width
\colseprulecolor{gray}

% title
\title{%
        \begin{tcolorbox}[before skip = \baselineskip, after skip =-\baselineskip]
            \centering\Huge\sffamily Cornell Notes on Something   
        \end{tcolorbox}
}

% ============================ Document attrs ==============================
\date{}
\parindent=0pt


% ============================ Document contents =========================== 

\begin{document}
    
\maketitle  

\clearpage

\tableofcontents

\clearpage

\section{Here we start the first topic}

\begin{paracol}{2}
    \question{This is a question.} 
    \note{The first piece of evidence is mandatory.}
    \note{Now add up to five\ldots}%
    \note{Here is some Python code: 
    }
\end{paracol}

\begin{pythoncode}[linenos=true,]{colback=base_c!5, colframe=base_c, title=My nice heading}
# indent your Python code to put into an email
import glob
# glob supports Unix style pathname extensions
python_files = glob.glob('*.py')
for file_name in sorted(python_files):
    print '    ------' + file_name

    with open(file_name) as f:
        for line in f:
            print '    ' + line.rstrip()

    print
\end{pythoncode}

\vspace*{3ex} % breaks the red line

\begin{paracol}{2}  
    \question{Here's another question.}
    \note{\lipsum[4]}
    \note{\lipsum[2]}   
\end{paracol}

\summary[\clearpage]{This is a longer box that will will close the question and {\Large \bfseries start a new page.} \lipsum[2]}

\section{Second topic}

\begin{paracol}{2}
    \question{Here's another question to begin the new page.}   
    \note{\lipsum[3]}
    \note{\lipsum[4]}
\end{paracol}
        
\summary{And another summary that will close the question and {\Large 
\bfseries stay in the same page.}}

\section{Another topic, several questions}

\begin{paracol}{2}
    \question{This is  question I.}     
    \note{I The first piece of evidence is mandatory.}
    \note{Now add up to five \ldots}%
    \note{\ldots\ additional pieces of evidence.}
\end{paracol}
\vspace*{3ex} % breaks the red line
    
\begin{paracol}{2}
    \question{Here's  question II.} 
    \note{II The first piece of evidence is mandatory.}
    \note{Now add up to five \ldots}%
    \note{\ldots\ additional pieces of evidence.}
\end{paracol}

\clearpage% breaks the red line
\begin{paracol}{2}
    \question{And question III.}
    
    \note{III The first piece of evidence is mandatory.}
    \note{Now add up to five\ldots}%
    \note{\ldots\ additional pieces of evidence.}
    \note{\ldots\ and more \ldots}
    \note{\ldots\ and more.}
\end{paracol}

\summary{And another summary that will close the question and {\Large \bfseries stay in the same page.}}        

\lipsum[1]
    
\end{document}