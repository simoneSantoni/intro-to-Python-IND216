% !TeX TXS-program:compile = txs:///pdflatex/[--shell-escape]
% =========================================================================
\documentclass[aspectratio=1610]{beamer}

% ========================= Theme =========================================
\usetheme{CambridgeUS}
\usecolortheme{default}

% ========================= Essential packages ============================
\usepackage{hyperref}

% ========================= Plotting ======================================
\usepackage{calc}
\usepackage{tikz}
\usetikzlibrary{arrows,
                arrows.meta,
                calc,
		chains,
                quotes,
                positioning,
		shapes,
                shapes.geometric}
\usepackage{graphicx}
\usepackage{graphics}
\usepackage{pgfplots}
\pgfplotsset{width=7cm,compat=1.17}

%% ============================== Tabular =================================
\usepackage{booktabs}
\usepackage{tabularx,ragged2e}
\usepackage{array}
\usepackage{multirow}
\usepackage{siunitx}
  \sisetup{detect-all}
\usepackage{adjustbox}
\usepackage{rotating}
\usepackage{threeparttable}
\usepackage[justification=centering]{caption}

%% ========================== Coding snippets =============================
% Default fixed font does not support bold face
\usepackage{minted}

% ========================= Infor on authors ==============================
\title{Intro to Python --- SMM692}
\subtitle{Python Objects}
\author{Simone Santoni}
\institute{Bayes Business School}
\date{MSc Pre-Course Series}

% ============================ Colors =====================================
\definecolor{base_c}{rgb}{0.6,0,0}
\definecolor{comp_c}{rgb}{0.09803921568627451, 0.6901960784313725, 0.7529411764705882}
\definecolor{tri_1}{rgb}{0.09803921568627451, 0.7686274509803922, 0.19215686274509805}
\definecolor{tri_2}{rgb}{0.19215686274509805, 0.09803921568627451, 0.7686274509803922}

% ========================= TOC  ==========================================
\AtBeginSubsection[]
{
    \begin{frame}
        \frametitle{Outline}
        \tableofcontents[currentsection,currentsubsection]
    \end{frame}
}

% ========================= Document  ====================================
\begin{document}

\begin{frame}
	\titlepage
\end{frame}

\begin{frame}{Outline}
	\tableofcontents
\end{frame}

% ------------------------- Background -----------------------------------

\section{The Chapter in a Nutshell}

\begin{frame}
    \frametitle{Scope}
    In Chapter 3, the attention revolves around the following topics:

    \begin{itemize}
        \item The concept of Python object 
        \item The types of Python objects
        \item The characteristics of each Python object
    \end{itemize}
\end{frame}

\begin{frame}
    \frametitle{Why Shall I Learn About Python Objects?}
    \begin{itemize}
    	\item Built-in objects make coding efficient and easy
        \begin{itemize}
            \item  For example, using the \href{https://docs.python.org/3/tutorial/introduction.html\#strings}{string} object, we can represent and manipulate a piece of text --- e.g., a newspaper article --- without loading any \href{https://docs.python.org/3/tutorial/modules.html}{module}
        \end{itemize}
           	\item Built-in objects are flexible
            \begin{itemize}
                \item For example, we can deploy built-in objects to create a \href{https://docs.python.org/3/tutorial/classes.html}{class}
            \end{itemize}
    	\item Built-in objects have been created and refined over time by a large community of expert developers. Hence, they are  often  more  efficient  than  ad-hoc objects (unless the creator of the ad-hoc object knows her business!)
    \end{itemize}
\end{frame}

\begin{frame}
    \frametitle{Learning Goals}
    At the end of the chapter, you will be able to evaluate the various types of Python objects regarding:
    \begin{itemize}
        \item Key features
        \item Use cases/roles
        \item Available methods
    \end{itemize}
\end{frame}

\section{Python Objects Fundamentals}

\begin{frame}
    \frametitle{What is a Python Object?}
    
    In essence, Python objects are pieces of data. Mark Lutz, the author of the popular book \href{https://www.google.co.uk/books/edition/Learning_Python/4pgQfXQvekcC?hl=en&gbpv=0}{\textcolor{blue}{Learning Python}}, points out

    \vspace{2em}
    
    \begin{quote}
    \textit{``... in Python, we do things with stuff. ``Things'' take the form of operations like addition and concatenation, and ``stuff'' refers to the objects on which we perform those operations''}
    \end{quote}

\end{frame}

\begin{frame}
    \frametitle{What Are the Main Families of Python Objects?}
    In Python, there are two families of objects:
    
    \begin{itemize}
        \item Built-in objects provided by the Python language itself
        \item Ad-hoc objects --- called \href{https://docs.python.org/3/tutorial/classes.html}{classes} --- we can create to accomplish specific goals
    \end{itemize}
    
    \end{frame}

\begin{frame}
    \frametitle{What Are the Main Types of Built-In Python Objects?}
    \begin{itemize}
        \item Strings
        \item Numbers
        \item Data containers
        \begin{itemize} 
            \item Lists 
            \item Dictionaries
            \item Tuples 
            \item Sets 
        \end{itemize}
        \item Files 
        \item Python statements, syntax, and control flow
        \item Iterators 
    \end{itemize}
\end{frame}

\section{Built-In Python Object Types }
\subsection{Numbers \& Strings}

\begin{frame}[fragile]{Integers and Floating Points}
	\begin{columns}[t]
		\begin{column}{0.45\textwidth}		
            
            The most common number types are integers and floating-point numbers:
            \begin{itemize}
                \item Integers are whole numbers such as 0, 4, or -12
                \item Floating-point numbers represent real numbers such as 0.5, 3.1415, or -1.6e-19
                \begin{itemize}
                    \item However, floating points in Python do not have --- in general
                    --- the same value as the real number they represent
                    \item It is worth noticing that any single number with a period `.'
                    is considered a floating point in Python 
                \end{itemize}
            \end{itemize}
	
    \end{column}
    
    \begin{column}{0.45\textwidth}
        
        Snippet 4.1 --- doing `stuff' with numbers
        \rule{\textwidth}{1pt}
        \scriptsize
        \begin{minted}{python}
# integer addition
>>> 1 + 1
2

# floating-point multiplication
>>> 10 * 0.5
5.0

# 3 to the power 100
>>> 3 ** 100
515377520732011331036461129765621272702107522001

        \end{minted}
        \rule{\textwidth}{1pt}
		
    \end{column}
\end{columns}
\end{frame}

\begin{frame}
    \frametitle{Number Type Objects in Python}
    \begin{table}[!htbp]
    	\centering
    	\caption*{Number Type Objects in Python}
    	\label{tab:number_types_in_python}
    	\begin{tabular}{ll}
    		\toprule \toprule
    		Literal & Interpretation\\
    		\midrule
    		 1234, -24, 0, 99999999999999 & Integers (unlimited size)\\
    		 1.23, 1., 3.14e-10, 4E210, 4.0e+210 & Floating-point numbers \\
    		 0o177, 0x9ff, 0b101010 & Octal, hex, and binary literals in 3.X \\
    		 0177, 0o177, 0x9ff, 0b101010 & Octal, octal, hex, and binary literals in 2.X \\
    		 3+4j, 3.0+4.0j, 3J & Complex number literals \\
    		 set(`spam'), \{1, 2, 3, 4\} & Sets: 2.X and 3.X construction forms\\ Decimal(`1.0'), Fraction(1, 3) & Decimal and fraction extension types\\
    		 bool(X), True, False & Boolean type and constants\\
    		 \bottomrule
    	\end{tabular}
    \end{table}
\end{frame}

\begin{frame}[fragile]
    \frametitle{String Type Fundamentals}
    \begin{columns}
        \begin{column}{0.35\textwidth}
            \quad \textbf{What:} A Python string is a positionally ordered collection of other
            objects. Strictly speaking, strings are \textit{immutable
            sequences} of one-character strings; other, more general sequence
            types include lists and tuples, covered later.

            \vspace{1em}
            
            \quad \textbf{How and why:} Strings are used to record words, contents of text files loaded into memory, Internet addresses, Python source code, and so on.
        \end{column}
        \begin{column}{0.60\textwidth}
        \normalsize Snippet 4.6 --- Python strings as sequences
        \rule{\textwidth}{1pt}
        \scriptsize
        \begin{minted}{python}
           
# the string
>>> S = "Python 3.X"
# check the length of S
>>> len(S)
6
# access the first unitary string in the sequence behind S 	
>>> S[0]
"P"
# access the last unitary string in the sequence behind S
>>> S[len(S)-1]
"X"
# or, equivalently
>>> S[-1]
"X"
# access the unitary strings between the i-th and j-th positions in the 
# sequence behind S
>>> S[2:5]
"tho"
        \end{minted}
        \rule{\textwidth}{1pt}
        \end{column}
    \end{columns}
\end{frame}

\begin{frame}
    \frametitle{Sting Literals and Operators}
    \begin{table}[!htbp]
    	\centering
    	\begin{tabular}{ll}
    		\toprule \toprule
    		Literal/operation & Interpretation \\
    		\midrule
    		\texttt{S = ""} &  Empty string \\
    		\texttt{S = `'} &   Single quotes, same as double quotes \\
    		\texttt{S = "spam's"} & Single quote as a string \\
    		\texttt{S = `spam$\setminus$'s'} & Escape symbol \\
    		\texttt{length(S)} & Length \\
    		\texttt{S[i]} & Index \\
    		\texttt{S[i:j]} & Slice \\
    		\texttt{S1 + S2} & Concatenate \\
    		\texttt{S * 3} & Repeat S $n$ times (e.g., three times)\\
    		\texttt{"text".join(strlist)} & Join multiple strings on a character (e.g., ``text'')\\
    		\texttt{"\{\}".format()} & String formatting expression \\ 
   		\bottomrule
    	\end{tabular}
    \end{table}
\end{frame}

\begin{frame}
    \frametitle{String Literals and Operators (cont'd)}	
    \begin{table}[!htbp]
        \centering
        \begin{tabular}{ll}
    		\toprule \toprule
    		Literal/operation & Interpretation \\
    		\midrule
     		\texttt{S.strip()} & Remove white spaces \\
    		\texttt{S.replace("pa", "xx")} & Replacement \\
    		\texttt{S.split(",")} & Split on a character (e.g., ``,'') \\
    		\texttt{S.lower()} & Case conversion --- to lower case \\
    		\texttt{S.upper()} & Case conversion --- to upper case \\
            \texttt{S.find("text")} & Search substring (e.g., "text") \\
    		\texttt{S.isdigit()} & Test if the string is a digit \\
    		\texttt{S.endswith("spam")} & End test \\
            \texttt{S.startswith("spam")} & Start test \\
    		\texttt{S = """...multiline..."""} & Triple-quoted block strings \\
    		\bottomrule
    	\end{tabular}
    \end{table}
\end{frame}

\subsection{Data Containers}

\begin{frame}[fragile]
    \frametitle{List Type Fundamentals}
\begin{columns}
    \begin{column}{0.35\textwidth}
    \quad \textbf{What:} A Python \href{https://docs.python.org/3/tutorial/datastructures.html}{\texttt{list}} is an \emph{ordered}, \emph{mutable} array of objects. A list is constructed by specifying the objects, separated by commas, between square brackets, \texttt{[]}
    
    \vspace{1em}

    \quad \textbf{How and why:} Lists are just places to collect other objects --- being numbers, strings, or even other lists --- so you can treat them as groups.
    \end{column}
    \begin{column}{0.6\textwidth}
    \normalsize Snippet 4.11 --- list indexing and slicing 
    \rule{\textwidth}{1pt}
        \scriptsize
        \begin{minted}{python}          
# the list
>>> L = [4, ["abc", 8.98]]
# get the first item of L
>>> L[0]
4
# get the second element of L
>>> L[1]
["abc", 8.98]
# get the first item of L's second item
>>> L[1][0]
"abc"
        \end{minted}
    \rule{\textwidth}{1pt}
    \end{column}
\end{columns}
\end{frame}

\begin{frame}
    \frametitle{Popular List Methods}
    \begin{table}[!htbp]
    	\centering
        \small
    	\begin{tabular}{lp{10cm}}
    		\toprule \toprule
    		Method & Synopsis \\
    		\midrule 
    		\texttt{L.append(X)} & Append an item to an existing list\\
    		\texttt{L.insert(i, X)} & Append an item to an existing list in position $i$ \\
    		\texttt{L.extend([X0, X1, X2])} & Extend an existing list with the items from another list\\
    		\texttt{L.index(X)} & Get the index of the first instance of the argument in an existing list\\
    		\texttt{L.count(X)} & Get the cardinality of an item in an existing list\\
    		\texttt{L.sort()} & Sort the items in an existing list\\
    		\texttt{L.reverse()} & Reverse the order of the items in an existing list\\
    		\texttt{L.copy()} & Get a copy of an existing list\\
    		\texttt{L.pop(i)} & Remove the item at the given position in the list, and return it\\
    		\texttt{L.remove(X)} & Remove the first instance of an item in an existing list\\
    		\texttt{L.clear()} & Remove all items in an existing list\\
    		\bottomrule 
    	\end{tabular}
    \end{table}
\end{frame}


\begin{frame}
    \frametitle{Dictionary Type Fundamentals}
\end{frame}

\begin{frame}
    \frametitle{Tuple Type Fundamentals}
\end{frame}

\begin{frame}
    \frametitle{Set Type Fundamentals}
\end{frame}

\subsection{Files}

\begin{frame}
    \frametitle{...}
\end{frame}

\subsection{Python Statements, Syntax, and Control Flow}

\begin{frame}
    \frametitle{...}
\end{frame}

\subsection{Iterators }

\begin{frame}
    \frametitle{...}
\end{frame}

\section{Wrap-Up}

\begin{frame}
    \frametitle{At the End of the Chapter, You Will Be Able to...}
\end{frame}

\end{document}